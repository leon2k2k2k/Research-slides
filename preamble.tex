\numberwithin{equation}{section}
\usepackage[utf8]{inputenc}
\usepackage[T1]{fontenc}
\usepackage{lmodern}
%\usepackage[mathscr]{euscript}
\usepackage[margin=1.33in]{geometry}
% \usepackage{setspace}
\setstretch{1.125}
\usepackage[usenames, dvipsnames]{xcolor}
% \usepackage{graphicx}
\usepackage{mathtools}
\usepackage{amssymb}
\usepackage{amsthm}
%\usepackage{xparse}
\usepackage{xspace}
\usepackage{tikz-cd}
\usepackage[backref=page, bookmarks=false]{hyperref}
% \usepackage[capitalize, noabbrev]{cleveref}
\setcounter{tocdepth}{1}

\newcommand{\mb}{\mathbf}

\newcommand{\iso}{\cong}
\newcommand{\isoto}{\xrightarrow{\simeq}}

\newcommand{\Z}{\mathbb Z}
\newcommand{\Q}{\mathbb Q}
\newcommand{\R}{\mathbb R}
\newcommand{\C}{\mathbb C}
\newcommand{\T}{\mathbb T}
\newcommand{\F}{F}
\newcommand{\ii}{\mathbf i}
\newcommand{\mcal}[1]{\mathcal{#1}}

\newcommand{\op}{\mathsf{op}}
\newcommand{\GL}{\mathrm{GL}}
\newcommand{\PSL}{\mathrm{PSL}}
% \newcommand{\Diff}{\mathrm{Diff}}
\newcommand{\Lie}{\mathrm{Lie}}
\newcommand{\Zn}{\Z(n)}
\newcommand{\Rn}{\R(n)}
\newcommand{\Bdot}{B_\bullet}
\newcommand{\Edot}{E_\bullet}
\newcommand{\BdotG}{B_\bullet G}
\newcommand{\EdotG}{E_{\bullet}G}
\newcommand{\BdotH}{B_\bullet H}
\newcommand{\BnablaG}{B_\nabla G}
\newcommand{\EnablaG}{E_\nabla G}
\newcommand{\BdotGamma}{B_\bullet \Gamma}
\newcommand{\GFG}{\Gamma \backslash F / G}
\newcommand{\GFR}{\Gamma \backslash F / \R}
\newcommand{\GF}{\Gamma \backslash F}
\newcommand{\Diff}{\mathrm{Diff}^+(S^1)}
\newcommand{\Mfld}{Mfld}
\newcommand{\Witt}{\mathfrak{w}}
\newcommand{\LieVir}{\mathfrak{vir}}
% \newcommand{\FGG}(F/G}

\newcommand{\point}{\ast}

\newcommand{\Zf}{\mb{Z}_2^F}
\newsavebox{\pullback}
\sbox\pullback{%
\begin{tikzpicture}%
\draw (0,0) -- (1ex,0ex);%
\draw (1ex,0ex) -- (1ex,1ex);%
\end{tikzpicture}}

\DeclareMathOperator{\Ext}{Ext}
\DeclareMathOperator{\Hom}{Hom}

\newtheorem{lemma}[equation]{Lemma}
\newtheorem{corollary}[equation]{Corollary}
\newtheorem{proposition}[equation]{Proposition}
\newtheorem{thm}[equation]{Theorem}
\newtheorem{theorem}[equation]{Theorem}
\newtheorem*{mainthm}{\cref{main_thm}}

\theoremstyle{definition}
\newtheorem{example}[equation]{Example}
\newtheorem{definition}[equation]{Definition}
\newtheorem{claim}[equation]{Claim}
\newtheorem{conj}[equation]{Conjecture}
\newtheorem{question}[equation]{Question}

\theoremstyle{remark}
\newtheorem{remark}[equation]{Remark}
\newtheorem*{fct}{Fact}
\newtheorem*{note}{Note}


\crefname{thm}{Theorem}{Theorems}
\crefname{lem}{Lemma}{Lemmas}
\crefname{cor}{Corollary}{Corollaries}
\crefname{prop}{Proposition}{Propositions}
\crefname{ex}{Exercise}{Exercises}
\crefname{exm}{Example}{Examples}
\crefname{defn}{Definition}{Definitions}
\crefname{claim}{Claim}{Claims}
\crefname{rem}{Remark}{Remarks}
\crefname{fct}{Fact}{Facts}
\crefname{note}{Note}{Notes}

\newcommand{\term}{\emph} % e.g. "The \term{trace} is defined to be..."

\DeclarePairedDelimiter\abs{\lvert}{\rvert}
\DeclarePairedDelimiter\set{\{}{\}}
\DeclarePairedDelimiter\paren{(}{)}
\DeclarePairedDelimiter\ang{\langle}{\rangle}
\DeclarePairedDelimiter\ket{\lvert}{\rangle}

\makeatletter
	\let\oldparen\paren
	\def\paren{\@ifstar{\oldparen}{\oldparen*}}
\makeatother

\usepackage{microtype}
\usepackage{hypcap}

\hypersetup{
 colorlinks,
 linkcolor={red!50!black},
 citecolor={green!50!black},
 urlcolor={blue!80!black}
}

% make cleveref use the Oxford comma:
% see https://tex.stackexchange.com/questions/161338
\newcommand{\creflastconjunction}{, and\nobreakspace}

\newcommand{\TODO}{\textcolor{red}{TODO}}

\newcommand{\cat}{\mathsf}
\newcommand{\Man}{\cat{Man}}
\newcommand{\Sh}{\cat{Sh}}
\newcommand{\Ch}{\cat{Ch}}
\newcommand{\sSet}{\cat{sSet}}

\newcommand{\Hsm}{H_{\mathrm{SM}}}

\newcommand{\Sym}{\mathrm{Sym}}
\newcommand{\fg}{\mathfrak g}
\newcommand{\gl}{\mathfrak{gl}}
\newcommand{\tr}{\mathrm{tr}}
\newcommand{\OmegaR}{\Omega^1 \otimes i\R}
\newcommand{\chern}{(c_1)^\nabla}


\renewcommand{\d}{\mathrm d}
\newcommand{\ud}{\,\d}

\newcommand{\tGam}{\widetilde\Gamma}
\newcommand{\CExt}{\mathrm{CExt}}
\newcommand{\Map}{\mathrm{Map}}

% was this deifned somewhere else...?
\newcommand{\AlgVir}{\mathfrak{vir}}
\newcommand{\GpVir}{\widetilde\Gamma}

% maintenance
\newcommand{\CW}[1]{\footnote{\textcolor[rgb]{0.60,0.00,0.50}{CW: #1}}}%
\newcommand{\AD}[1]{\footnote{\textcolor[rgb]{0.00,0.50,0.50}{AD: #1}}}%
\newcommand{\YL}[1]{\footnote{\color{orange} {\bf YL:} {#1}}}

\newcommand{\IZ}{I\mb{Z}}